\documentclass[a4paper,10pt]{article}
\usepackage[utf8]{inputenc}

%opening
\title{Mi primer documento Latex} % Cambia el titulo por alguno que tu elijas.
\author{Janet Martinez Garcia} % Cambialo por tu nombre completo.

\begin{document}

\maketitle

% \begin{abstract}
% 
% \end{abstract}

\section{Seccion 1}

Hola profes.\\ % Cambia este texto por algo tuyo.

Esta es mi primer tarea en Latex y me emociona poder aprender a utilizarlo.\\% Cambia este texto por algun texto tuyo.

$ e^{i \pi} + 1 = 0$\\
La formula de Euler, es considerada la mas bella del mundo, por que en una forma tan compacta de relacionarse con las cinco entidades fundamentales de las matematicas\\ “Esta identidad se puede utilizar para obtener coordenadas con unos cálculos ridículos y una precisión asombrosa sin necesidad de calcular senos y cosenos, fue el primer paso (o uno de los primeros) para los sistemas GPS y GPRS que utilizan las coordenadas polares para determinar la posición del dispositivo y permitir su comunicación.
Y la belleza de todas estas cuestiones es la misma que pueda tener una obra de Shakespeare o una piesa de Mosart “\\

$x = \frac {-b \pm \sqrt {b^2 - 4ac}}{2a}$\\
Esta es la formula general, mas conocida como la “CHICHARRONERA“\\

  % Comenta esta ecacución y escribe abajo una ecuación muy simple (la que quieras).

\end{document}
